%%% Laboratory	 Notes
%%% Template by Mikhail Klassen, April 2013
%%% Contributions from Sarah Mount, May 2014
\documentclass[a4paper]{tufte-handout}

\newcommand{\workingDate}{\textsc{Dec $|$ 2016}}
\newcommand{\userName}{Ezeddin Al Hakim}
\newcommand{\institution}{KTH}

\usepackage{lab_notes}

\usepackage{hyperref}
\hypersetup{
    pdffitwindow=false,            % window fit to page
    pdfstartview={Fit},            % fits width of page to window
    pdftitle={Projekt lab notes 2016},     % document title
    pdfauthor={Ezeddin Al Hakim},         % author name
    pdfsubject={Ezeddin Al Hakim},                 % document topic(s)
    pdfnewwindow=true,             % links in new window
    colorlinks=true,               % coloured links, not boxed
    linkcolor=DarkScarletRed,      % colour of internal links
    citecolor=DarkChameleon,       % colour of links to bibliography
    filecolor=DarkPlum,            % colour of file links
    urlcolor=DarkSkyBlue           % colour of external links
}


\title{Projekt lab notes by Ezeddin Al Hakim}
\date{2016}

\begin{document}
\maketitle

%%%%%%%%%%%%%%%%%%%%%%%%%%%%%%%%%%%%%%%%%%%%%%%%%%%%%%%%

\begin{projects}
	\begin{description}
		\item Classifier for identification of signal peptides.
    \end{description}
\end{projects}

%%%%%%%%%%%%%%%%%%%%%%%%%%%%%%%%%%%%%%%%%%%%%%%%%%%%%%%%
\newday{19 December 2016}

We have described the classification problem by a Hidden Markov Model. In the next days, we will focus to implement HMM, whether we have time we will implement RNN.

\newday{20 December 2016}
We have installed \textit{hmmlearn} module, it is a set of algorithms for inference of Hidden Markov Models for Python. Then we implemented the first version of the classifier using hmmlearn. The classifier works as follows:
\begin{itemize}
\item Training a model $\lambda_p$ using only positive data 
\item Training a model $\lambda_n$ using only negative data 
\item   Given a sequence of amino acids $X_{1:n}$, we will calculate the probability for both models (i.e. $P(X_{1:n}|\lambda_p)$ and $P(X_{1:n}|\lambda_n)$ ) and choosing the model with highest probability.
\end{itemize}

But classifier was not so good, we got 61 \% accuracy,  100 \% recal and 58 \% precission. 

\newday{21 december 2016}
Today we have tried another method. We have trained a model $\lambda_a$  using all data, positive and negative data. To classify signal peptides, we calculate/find the most likely sequence of hidden states $Z_{1:n}$ ( n-, h-, c-, C- and other-regions), i.e. maximizes the conditional distribution  $P(Z_{1:n}|X_{1:n},\lambda_a)$. To classify if the sequence is a signal peptide, we finding the n-, h-, c-, and C-regions in the hidden states.

The classifier was much better, we got 92 \% accuracy,  92 \% recal and 91 \% precission. 

\newday{22 December 2016}

Today we tried to evaluate our classifier on two proteom, human and mouse. But we had a few bugs in our classifier, e.g. it was unknown characters in the sequences of  the two proteom.



\newday{27 December 2016}

Today we succeeded to evaluate our classifier on the two proteom, and we got the follows results:
\\
-------------------\\
HUMAN\\
All 215929\\
All true 12931\\
All positive 17181\\
True positive 10626\\
True negative 196443\\
Precission 0.618473895582\\
Recal 0.821746191323\\
Accuracy on test data: 95.9\%\\
-------------------\\
MOUSE\\
All 124168\\
All true 7756\\
All positive 9966\\
True positive 6246\\
True negative 112692\\
Precission 0.626730885009\\
Recal 0.805312016503\\
Accuracy on test data: 95.79\%\\
-------------------\\



\newday{28 December 2016}

Today we tested our classifier with different length of random sequences and 92 \% of them classified as  non-signal peptides. Here are some outputs from Controls:
\\
-------------------\\
Evaluated on random data with length 2:\\
All true 0\\
All positive 0\\
True positive 0\\
True negative 1000\\
Accuracy on test data: 100.0\%\\
-------------------\\
Evaluated on random data with length 100:\\
All true 0 \\
All positive 110\\
True positive 0\\
True negative 890\\
Accuracy on test data: 89.0\% \\
-------------------\\
Evaluated on random data with length 10000:\\
All true 0\\
All positive 118\\
True positive 0\\
True negative 882\\
Accuracy on test data: 88.2\%\\
-------------------\\

\hrulefill

%%%%%%%%%%%%%%%%%%%%%%%%%%%%%%%%%%%%%%%%%%%%%%%%%%%%%%%%
\bibliographystyle{plain}
\bibliography{lab_notes}

\end{document}
